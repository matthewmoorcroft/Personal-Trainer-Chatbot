\chapter{Introduction}\label{chap:1}
The main goal of this final degree project is to develop an application or controller based on the SDN \cite{sdn_wiki} paradigm capable of implementing distributed mobility management (DMM) \cite{ddm_standars_landscape} within the context of the CROWD EU FP7 project \footnote{\url{http://www.ict-crowd.eu/}}. Taking the problem of ultra dense future networks into consideration, Distributed Mobility Management (DMM) working group (WG) of the IETF \footnote{\url{http://datatracker.ietf.org/wg/dmm/charter/}} has presented different solutions to solve the mobility issue. The main idea is to swap from current hierarchical network architectures to another paradigm based on decentralized entities.\\

To achieve these mobility objectives we will use the approach of SDN due to its benefits: low economic costs, high tunning capabilities and separation of control and data planes. A further description of the possibilities brought by the use of SDN can be found on the State of the Art Chapter \ref{sec:chap2_sdn}. \\

The communications protocol selected that gives access to the forwarding plane of a network switch over the network is OpenFlow\cite{openflow_white_paper}. OpenFlow allows remote administration of a switch's packet forwarding tables, by adding, modifying and removing packet matching rules and actions. Thanks to that, routing decisions can be made ad hoc by the controller of the network. Packets which are unmatched by the switch are forwarded to the controller and depending of the type, source and destination of each packet, the controller decides to modify, install or delete existing flow table rules on one or more OpenFlow controlled switches. \\

There are several Open Flow controllers available to manage the approach set out by the DMM study group. For this project we have chosen Ryu as an Open Flow controller due to its advantages discussed on the Solution Proposed, Chapter \ref{chap:3}. The software will manage all the connections and traffic through an IPv6 \cite{ipv6} network. We will design an implementation of the scenario envisioned in the CROWD project with two districts managed by a CROWD Regional Controller and internally managed by a CROWD Local Controller. Once we have decided the network design we will implement a testbed to test the performance of the mobility management solution. This testbed will include some access points, open flow nodes, openflow controllers and a mobile node capable to perform different mobility handovers focused on assessing the viability solution proposed by the DMM study group. Furthermore, the mobile node will be able to perform a handover from our IPv6 network to a different IPv4 \cite{ipv4} network. \\

Finally on the Experiments (Chapter \ref{chap:5}) we will do different tests to completely evaluate the performance of the SDN deployed and to check if the present solution is capable of managing the mobility of users and their performance.

\section{Analysis of the problem}\label{sec:chap1_analysis_of_the_problem}
Current networks architectures are deployed in a hierarchical manner, relying on a centralized gateway. Thus, the existing IP mobility management protocols are generally deployed in a centralized manner. All the data traffic passes through a centralized mobility anchor, such as the HA in MIPv6 \cite{mipv6} or the LMA in PMIPv6 \cite{pmipv6}, and all the bindings are managed at this anchor as well. As the number of MNs and the volume of the mobile data traffic increase, such centralized architectures may encounter scalability issues (e.g. network bottlenecks, and single point of failure), security issues (e.g. attacks focused on the centralized anchor), as well as performance issues (e.g. centralized and non-optimal routing). In addition, existing IP mobility protocols are designed to be always activated, managing all the services and all the traffic in the same way. They do not take into consideration that a given MN may not move during the use of a service (which is 60\% of the cases in operational networks) or that a service may not require mobility functions at all. Such approaches may thus lead to non-optimal routing and large overhead due to tunnelling mechanisms.\\

In order to cope with the rapid traffic explosion we are witnessing, IP mobility management protocols need to be adapted for such evolution. There is a need to define novel mobility management mechanisms that are both distributed and offered dynamically. They should be distributed in order to avoid any network bottleneck or single point of failure, and to provide better reliability. They should be activated/deactivated dynamically as needed, in order to globally reduce their signalling load and to increase the achieved performance.
Accordingly, the IETF chartered recently the DMM working group in 2012. Various efforts from both industry and academia are being performed on specifying DMM schemes. A common feature between different DMM schemes is distributing the mobility anchoring at the AR level. The MN changes dynamically its mobility anchor for new sessions, while keeping the previous anchors of ongoing sessions. When the sessions anchored at a specific mobility anchor are terminated, the MN unregisters from that anchor. Assuming that most of the sessions are relatively short, most of the data traffic is routed optimally without tunnelling.
One of the DMM requirements is to rely on the existing IP mobility protocols by extending and adapting them. This is in order to benefit such standardized protocols before specifying new ones, and also to facilitate the migration of networks architectures. \\

Thus hereafter, we will consider two main approaches. The first is MIPv6-based \cite{mipv6}, providing global as well as local mobility support for MNs that may move between several access networks. This protocol  is designed to allow mobile device users to move from one network to another while maintaining a permanent IP address, allowing nodes to remain reachable while moving around in the IPv6 Internet. To support these operations, Mobile IPv6 defines a new IPv6 protocol and a new destination option.  All IPv6 nodes, whether mobile or stationary, can communicate with mobile nodes. The second protocol is PMIPv6-based \cite{pmipv6}, providing local mobility support for MNs moving in a single operational domain. It is a protocol for building a common and access technology independent of mobile core networks. Proxy Mobile IPv6 is the only network-based mobility management protocol standardized by IETF. A further explanation of protocols MIPv6 \ref{sec:chap2_dmm_mipv6} and PMIPv6 \ref{sec:chap2_dmm_pmipv6} can be found on the State of the Art (Chapter \ref{chap:2})